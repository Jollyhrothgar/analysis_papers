\chapter{Physics Background}
\section{The Phenomena of Spin}

Spin is a fundamental quantity possessed by all elementary particles. We use
the word 'spin' to describe the property, because partices which possess spin,
behave as though they have some kind of intrinsic, hidden rotation, as if they
were 'spinning'. The dimension of spin, therefore is angular momentum. What is
somewhat bizarre about spin, is that we do not observe anything physically
spinning - although there are some phenomena (such as oribtal angular momenta)
which can be naively thought of as a 'spinning system' (but this description
escapes classical analogy, due to its quantum, probibalistic nature). The role
of Spin in Physics is of foundational importance, and yet, we have not
succesfully produced a model which can accurately predict the spin of hadrons.

The presence of spin in relativistic particles creates the phenomona of
chriality, which has huge implications for how elementary particles can generate
structure in matter itself ~\needcite{}. In the case of the weak interaction,
the presence of spin, which creates Chiral spinors breaks the left-right
symmetry of weak coupling in matter (a fact which will be exploited in this
thesis to probe the spin of the proton sea).

The phenomena of spin also changes the rules for how ensembles of particles may
exist in a potential. Particles with spin are fermions, and because these
paritcles must obey fermi statistics, we can observe structure in matter in the
universe ~\needcite{}. Without spin, the world as we know would collapse on
itself, making any kind of extended non-exotic structures which currenlty exist
by virtue of the Pauli exclusion principal, impossible.

\section{A Brief History of Proton Spin}

The study of Spin is really just an outgrowth of the general study of matter.
Our models for matter, and the underlying structure of matter (in the modern
sense), represents over a hundred years of experimental and theoretical efforts.
The fundamental tool for this study is scattering. Scattering offers a very
powerful method where we one uses a well known initial state of matter
(typically in the form of a beam), allows this beam to interact with an unknown
configuration of matter, and measures the scattered beam. By carefully studying
the kinematics of the scattered beam, we can create models which allow us to
understand the structure of the target matter. We may also use these models to
understand the interaction between the beam, and the target. 

Although Gell-Mann's simple quark model of baryons ~\needcite{} predicts the
correct quantiy for the spin of the proton, the work of Ashman et al (1988)
~\needcite{} at the European Muon Collaboraiton directly measured a portion of
the proton sturcture function $g_1$ and found that a rather small fraction of
the prton spin comes from quarks - and most of the spin is carried by the
gluons (Figure~\ref{fig:emc_g1_result}). 

\begin{figure}[H]
	\begin{center}
	\includegraphics[width=0.5\linewidth]{../filler/squareimg.png}
	\caption{~\needfig{} ~\needcap{}. Results of EMC experiment showing that the structure
	function g1, tells us a thing about proton spin.}
	\label{fig:emc_g1_result}
\end{center}
\end{figure}

\section{Proton Spin Crisis}
\section{How to Model Proton Spin}
\begin{itemize}
		\item structure functions
		\item proton spin decomposition
		\item unpolarized parton distribution functions
		\item polarized parton distribution functions
		\item that sweet table from Delia hasch
		\item discussion $\bar{q}$, $q$, $L_q$, $g$
		\item DSSV figures
\end{itemize}
\section{How to Measure Proton Spin}
\begin{itemize}
		\item physics probes for proton spin
		\item W cross section
		\item derivation of Asymmetry
		\item kinematic extremes of Asymmetry
\end{itemize}
\subsection{Past Experimental Efforts}
\begin{itemize}
		\item summary of data on structure functions
		\item fixed target experiments
		\item collider experiments
\end{itemize}

\section{World Efforts to Measure Proton Spin}
\subsection{CERN}
\subsubsection{ZEUS}
\subsubsection{HERA}
\subsubsection{HERMES}
\subsubsection{COMPASS}
\subsubsection{EMC}
\subsection{SLAC}
\subsection{JLAB}

\section{How to Measure Beam Luminosity in Collider Experiments}
\begin{itemize}
		\item vernier analysis note intro, equations
		\item summarize the papers on Lumoninosity
\end{itemize}
