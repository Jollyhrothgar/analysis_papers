\clearpage
\chapter{Multiple Collisions}
\label{ch:MultipleCollisions}

When more than one collision occurs during a bunch crossing, the BBC will
undercount the collision rate. This will lower our overall calculated
luminosity, and must be corrected. The BBC is capable of recording either 0 or 1
collisions in a bunch crossing, but sometimes, there are more.

The process of generating an event from two colliding bunches follows Poisson
statistics, which characterize the probability distribution of an event which
can either happen, or not. Here, we follow the notational convention
of~\cite{Wolin2016}, and summarize the relevant portions of the work here. We
define the `true' BBC rate as a parameter, $\mu$. For any collision at the
PHENIX IR, we restrict the total number of possible scattering interactions to
the following:

\begin{itemize}
  \item A collision occurs, but there are no hits in the BBC north or BBC south.
    We can characterize the probability of this event with $\epsilon_0$.
  \item A collision occurs, but only the BBC south registers a hit. We
    characterize this probability with $\epsilon_S$. 
  \item A collision occurs, but only the BBC north registers a hit. We
    characterize this probability with $\epsilon_N$. 
  \item A collision occurs and is registered in both BBCs. This probability is
    characterized with $\epsilon_{NS}$.
\end{itemize}

Given that we have restricted these four outcomes to be the only possible
outcomes, their probability must sum to unity.

\begin{equation}
  \epsilon_0 + \epsilon_S + \epsilon_N + \epsilon_{NS} = 1
  \label{eq:bbc_collision_probabilty}
\end{equation}

Though Wolin includes a discussion of parameterizing the event collision
probabilities into further subdivisions, which are physically motivated, but not
well defined in terms of observables at PHENIX, I stick with the convention of
using only what we may observe at PHENIX.

Proceeding along with Wolin's work, we introduce two new probabilities:

\begin{itemize}
  \item $P_{kl}$ - The probability characterizing the condition when the north
    BBC is hit by $k$ different collisions, and the south BBC is kit by $l$
    different collisions.
  \item $P(kl\vert N)$ - The probability of seeing $k$ collisions in the
    north BBC and $l$ collisions in the south BBC, given that $N$ collisions
    have occurred.
\end{itemize}

A caveat of the BBC and ZDC detectors is that, due to design limitations and/or
practice, the detectors do not resolve individual hit times in the case where
there are more than one collision.

As an exercise, we can observe the process of calculating the simplest case for
$P_{kl}$, which is when no events were observed in the north, or south BBCs,
i.e., $P_{00}$. Recall that we wish to recover the true number of collisions,
$N$. Therefore, $P_{00}$ is written as the sum of all conditional
probabilities, $P(00|N)$ which satisfy $P_{kl} = P_{00}$:

\begin{equation}
  P_{00} = \sum_{i=0}^{\infty} P(00\vert i)P(i,\mu)
  \label{eq:p_simple}
\end{equation}

That is to say, that any number of collisions may result in no detected
collisions, so we must sum over all possible probabilities to obtain the true
probability of i = N events producing no collisions. $P(i,\mu)$ is the Poisson
probability distribution which represents the likelihood of i independent events
occurring given an average event rate $\mu$. It is worth noting the rarity of pp
collisions, when one considers the magnitude of the number of protons in two
bunches - $\mathcal{O}(10^{11})$ compared to the order of collisions per bunch
crossing - $\mathcal{O}(1)$.

\edithere{}
