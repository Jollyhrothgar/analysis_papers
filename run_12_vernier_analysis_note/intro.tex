\pagestyle{empty}
\renewcommand*\familydefault{\sfdefault}
{\sffamily
\vspace{2cm}
%\centerline{\HUGE The PHENIX Experiment at RHIC}

\vspace{1.5cm}


\vspace{0.5cm}
\centerline{\huge \bf{Vernier Scan Analysis}}
\vspace{0.25cm}
\centerline{\emph{Determining Absolute Luminosity Delivered by RHIC}}
\centerline{\emph{Run 12 analysis of 500 $GeV$ and 200 $GeV$ $p+p$ collisions}}

\vfill

\centerline{\Large Brookhaven National Laboratory}

\vspace{0.5cm}

\centerline{\Large 03 September 2015}

\vfill
}

\begin{figure}[H]
  \begin{center}
    \includegraphics[width=0.8\linewidth]{{./figures/scan_symbol_no_axes}.eps}
  \end{center}
\end{figure}

\vfill

\hspace*{0.2in}\emph {University of California, Riverside} \ 
               \hspace{0.25in} {\bf Mike Beaumier}, {\bf Ken Barish}, {\bf Richard Hollis}

\hspace*{0.2in}\emph {STAR Experiment} \ 
               \hspace{0.25in} {\bf K. Oleg Eyser} \
\renewcommand*\familydefault{\rmdefault}



\clearpage

\pagestyle{fancy}

%==============================================  Excecutive Summary
\chapter{Introduction and Executive Summary}
\label{ch:intro}

\textcolor{red}{\textbf{NOTE THAT THIS IS A DRAFT, AND SOME SECTIONS EXIST AS MY
ANNOTATIONS OF PUBLISHED WORK, NOT MY OWN. THE FINAL VERSION WILL BE PROPERLY
QUOTED, CITED, ETC}}

The Vernier Scan Analysis is typically done every year, its purpose is to
calculate the absolute luminosity of collisions delivered to PHENIX's
interaction region (IR) by RHIC.  Absolute luminosity is a necessary for the
normalization of any cross-section (but not necessary for cross-section ratos).  

The vernier analysis is done on a very special data set - obtained during a
vernier scan.  Unlike normal data taking, PHENIX must use a special trigger
configuration optimized for recording very low rates of minimum bias data. The
vernier scan itself consists of either the blue, or yellow beams, being
incrementally scanned across the other stationary beam.  If we assume that the
blue and yellow beams are identical, this scan will effecitively show us the
transverse dimensions of the blue and yellow beams - these dimenions are used to
calculate the luminosity, $\mathcal{L}$, and to verify that our model for
relativistic beam collisions is correct.

We can also use the vernier scan to calibrate the Beam Beam Counters (BBC) in
order to use them as luminosity monitors. For any BBC trigger, we can represent
the relationship between $\mathcal{L}$ and the cross section observed by the BBC
as:

\begin{equation} 
\label{eq:lumi_xsec_simple} 
\mathcal{L}_{BBC} = {R_{BBC}\over\sigma_{BBC}} 
\end{equation}

Where $\mathcal{L}_{BBC}$ is the effective luminosity delivered to a specific
BBC trigger, $R_{BBC}$ is the live event rate of the BBC trigger, and
$\sigma_{BBC}$ is defined as the cumulative cross section of events measured by
this trigger.

The absolute luminosity is calculated as:

\begin{equation} 
\label{eq:lumi_one_bunch} 
\mathcal{L} = {f_{bunch}N_{b} N_{y}\over{2\pi\sigma_{x}\sigma_{y}}} 
\end{equation}

Where $\mathcal{L}$ is the luminosity, $f_{bunch}$ is the frequency of each
individual bunch crossing, $N_{b}, N_{y}$ are the bunch populations for the
specific blue and yellow beams, resepectively, and $\sigma_{x}, \sigma_{y}$ are
the transverse widths both bunches in the x and y directions. We assume
identical beam bunch distributions for the blue and yellow beams
~\cite{AN888Datta2010}, however individual bunches can also be studied - and
this analysis is presented in this note. $f_{bunch}$ corresponds to the bunch
crossing rate of a single specific bunch, and depends on the number of bunches
in a specific fill and the blue beam clock.

For clarity: The blue beam clock ticks once every time there is a bunch
crossing.  Therefore, for $120$ bunch fills (including filled and empty
bunches), and the standard blue beam clock frequency, $f_{clock}$, of $9.36
MHz$, $f_{bunch} \equiv f_{clock} / 120 = 78 kHz$.

Previous analysis notes on the vernier analysis are listed below for
benefit/curiosity of the reader:

\begin{itemize}
\item AN184~\cite{AN184Belikov2003}
\item AN597~\cite{an597Bazilevsky2007}
\item AN688~\cite{an688Bennet2008}
\item AN888~\cite{AN888Datta2010}
\end{itemize}

Global vernier scan characteristics are summarized in table
~\ref{tab:global_scan_summary}. Each vernier scan is done slightly differently
as a systematic check - beam scan order is varied, beam energy is varied, scan
length and step length is varied, and scanning patterns are varied. We expect
that these variations will not affect our final result.

\begin{sidewaystable}[ht]
\centering
\begin{tabular}{ccccccccc}
\toprule
\textbf{Run}    & \textbf{Fill}   & \textbf{Energy}          & \textbf{Scan}       & \textbf{Scan}    & \textbf{Scan}  & \textbf{Beam}    &                & \textbf{Step}     \\
\textbf{Number} & \textbf{Number} & \textbf{($GeV\sqrt{s}$)} & \textbf{Time (min)} & \textbf{Pattern} & \textbf{Order} & \textbf{Scanned} & \textbf{Steps} & \textbf{Time (s)} \\
\midrule
359711 & 16444 & 200 & 41 & Type 1 & H - V & Blue  & 26 & 57.5 \\
360879 & 16470 & 200 & 41 & Type 1 & H - V & Yellow& 26 & 61.2 \\
362492 & 16514 & 200 & 50 & Type 1 & V - H & Blue  & 26 & 62.3 \\
364636 & 16587 & 510 & 58 & Type 2 & H - V & Yellow& 18 & 21.7 \\
365866 & 16625 & 510 & 53 & Type 1 & H - V & Blue  & 26 & 70.0 \\
366605 & 16655 & 510 & 54 & Type 1 & H - V & Yellow& 26 & 67.7 \\
367138 & 16671 & 510 & 54 & Type 1 & H - V & Blue  & 26 & 68.65\\
\bottomrule
\end{tabular}
\caption{ A summary of vernier scans in run 12 }
\label{tab:global_scan_summary}
\end{sidewaystable}

\begin{figure}[ht]
\begin{center}
\includegraphics[width=0.75\linewidth]{./figures/scan_patterns}
\caption{ Two different scanning patterns were used in Run 12. Type 2 was previously
used in past years prior to 2012, and is included in this year's set of scans as a consistency check.
Type 1 was used for the majority of the scans in 2012. The scan order introduces
systematic effects based on rate losses in the beam width calculation, which are described
later.}
\label{fig:scan_patterns}
\end{center}
\end{figure}

\setcounter{page}{1}

\clearpage

\resetlinenumber

\clearpage

\resetlinenumber

\tableofcontents

\clearpage

\resetlinenumber
