
%%%%%%%%%%%%%%%%%%%%%%%%%%%%%%%%%%%%%%%%%%%%%%%%%%%%%%%%%%%%%%%%%
%   Chapter: Acknowledgements 
%   Tue Sep 22 15:59:26 EDT 2015 %
%%%%%%%%%%%%%%%%%%%%%%%%%%%%%%%%%%%%%%%%%%%%%%%%%%%%%%%%%%%%%%%%%

\chapter{Acknowledgements} 
\label{ch:Acknowledgements}

Though I am sure I will miss people, I feel it is important to at least try to thank those
who have significantly contributed to this analysis, and my development as a graduate
student, because I was a fresh, green grad student when Is tarted this analysis, and am
now nearing the completion of my PhD. So, finishing this analysis is quite personally
significant, for me. 

I thank my adviser, Ken Barish, who has financially supported me throughout graduate
school, and allowed me to fully pursue research with few distractions. He has put me into
contact with experts which have helped me immeasurably, and given me lots of insight as to
the inner mechanics of how a large collaboration works, how to manage graduate students
and run a research group, and has given me plenty of freedom to develop personally and
professionally according to my interests, without sacrificing guidance along the way
towards completing my analyses and PhD.

I would like to thank Oleg Eyser, who I'm sure could find better uses for his time than
teaching fresh graduate students the basics of C++ programming and root, but still made
sure I was on track early in my work on this analysis.

I would like to thank Richard Hollis, who has been kind, patient, and good spirited in
supporting the many students of my group in their work and struggles with tough problems.
Richard has always managed to find time to look over somebody's code, prepare brief
examples, and tools for more efficient analysis.

I'd also like to thank Joe Seele, Martin Purshke, John LeJoie, Chris Pinkenburg and John
Koster, for their technical assistance. Martin was instrumental in helping to solve the
issue of recovering time-stamps which allowed us to time-order this analysis, using scaler
events to calculate live-time for various triggers, and many other nitty gritty details
which only an expert at PHENIX software can help with. Joe was very patient with many of
my questions regarding learning how to be a better programmer, without his help, I would
still be running ROOT macros, instead of creating efficient, fast software libraries,
extending my reach beyond just ROOT. Joe's help has allowed my analyses to become flexible
and portable, which I am very grateful for. John LeJoie has helped me understand the
subtle differences between various types of trigger scalers, the concept of a trigger to
begin with, and given me other guidance through the inner-workings of the triggering
portion of the PHENIX DAQ. Chris and John Koster's dedication to building excellent
software, and their mentorship of me during my tenure as fast production assistant in 2012
has had rippling affects on my analysis style - without them, I would not have developed
an intuition for how to parallelize my software schemes for running on computing clusters,
or learned powerful supporting tools such as regular expressions, and Perl scripting.

I'd also like to thank Angelika Drees, Kathy DeBlasio, Gregory Ottino and Amaresh
Datta, whose direct collaboration with regards to the Vernier Scans themselves, and in the
analysis of Vernier Scan data sets has provided me with great feed back and support, as
well as some cross checks.

Thanks too, to the Spin PWG conveners, whose questions and guidance during Spin PWG
meetings always seem to ferret out problems in analyses and point analyzers in the right
direction. Finally, there have been many, many graduate students and post docs who have
mentored me during my time at PHENIX. There are a thousand little things to learn about
what feels like a thousand different subsystems, and I would have been lost were it not
for those friendly people pointing me in the right direction.  So, if I've ever asked you
a question, you have my heartfelt thanks for answering me.
