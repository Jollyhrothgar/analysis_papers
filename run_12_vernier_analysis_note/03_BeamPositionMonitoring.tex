
%%%%%%%%%%%%%%%%%%%%%%%%%%%%%%%%%%%%%%%%%%%%%%%%%%%%%%%%%%%%%%%%%
%                                                               %
% Chapter: BeamPositionMonitoring Thu Sep 24 16:02:10 EDT 2015
%                                                               %
%%%%%%%%%%%%%%%%%%%%%%%%%%%%%%%%%%%%%%%%%%%%%%%%%%%%%%%%%%%%%%%%%

\chapter{Beam Position Monitoring}
\label{ch:BeamPositionMonitoring}

The Beam Position Monitors (BPMs) have played an important role in this
analysis, and in past analyses. They are the one means by which we can directly
measure the approximate transverse location of the beams. There are two sets of
BPMs on either side of the PHENIX IR, which we call "sector 7" and "sector 8".
From these measurements, we can reconstrcut the separation of the blue and
yellow beams during a vernier scan. Our purpose in doing so, is to correlate the
beam displacement, to the event rate measured by the BBCs, in order to directly
measure the width of the beams in the x and y directions (assuming the PHENIX
standard coordinate system) \textcolor{red}{\textbf{missing-fig phenix
coordinate system}}. CAD executes these beam displacements through using DX 
magnets which deflect the beams. The beams follow a planned presciption of
movement, and are stepped into, and out of alignment while PHENIX records data.

Past analyses have assumed that the data from the BPMS are unreliable. At times,
the polarity of the BPM measurement is flipped from the polarity of the planned
step.  Regardless, we can use the BPM measurement to obtain a set of beam
displacements, and correlate those displacements with time, and with the BBC
Rates using PRDF data, so the utility is not insignificant. We can also identify
which beam is scanned first from the BPM data and the timing of each step. We
can simply substitute planned step displacements if we don't trust the value of
beam displacements. We can see summarized the difference betweeen planned steps,
and real steps summarized in tables
\textcolor{red}{\textbf{missing-tab plannedsteps}}

\begin{table}
\centering
\begin{tabular}{c c c c c c c c}
\toprule
\textbf{$CAD_{x}$} & \textbf{$BPM_{x}$ } & \textbf{$\Delta_{x}$} &\textbf{$CAD_{y}$} & \textbf{$BPM_{y}$} & \textbf{$\Delta_{y}$} & \textbf{$BPM_{tot}$} & \textbf{$\Delta_{tot}$} \\
($10^{-5} m$) & ($10^{-5} m$) & (\%diff) & ($10^{-5} m$) & ($10^{-5} m$) & (\%diff) & ($10^{-5} m$) & (\%diff) \\
\midrule
-1000.00 & -1103.61 &  (10 \%) & 0.00 & 8.07 &  & 1103.64 &  (10 \%)\\
-750.00 & -854.70 &  (14 \%) & 0.00 & 10.67 &  & 854.77 &  (14 \%)\\
-600.00 & -720.00 &  (20 \%) & 0.00 & 11.86 &  & 720.10 &  (20 \%)\\
-450.00 & -567.61 &  (26 \%) & 0.00 & 13.04 &  & 567.76 &  (26 \%)\\
-300.00 & -419.25 &  (40 \%) & 0.00 & 14.25 &  & 419.49 &  (40 \%)\\
-150.00 & -258.89 &  (73 \%) & 0.00 & 16.00 &  & 259.39 &  (73 \%)\\
0.00 & -106.14 &  & 0.00 & 17.32 &  & 107.55 & \\
150.00 & 63.04 &  (58 \%) & 0.00 & 20.89 &  & 66.41 &  (56 \%)\\
300.00 & 227.11 &  (24 \%) & 0.00 & 21.36 &  & 228.11 &  (24 \%)\\
450.00 & 391.46 &  (13 \%) & 0.00 & 24.82 &  & 392.25 &  (13 \%)\\
600.00 & 542.27 &  (10 \%) & 0.00 & 28.10 &  & 542.99 &  (10 \%)\\
750.00 & 689.36 &  (8 \%) & 0.00 & 30.36 &  & 690.03 &  (8 \%)\\
1000.00 & 940.39 &  (6 \%) & 0.00 & 34.04 &  & 941.01 &  (6 \%)\\
0.00 & -93.96 &  & -1000.00 & -1017.07 &  (2 \%) & 1021.40 &  (2 \%)\\
0.00 & -94.75 &  & -750.00 & -767.29 &  (2 \%) & 773.11 &  (3 \%)\\
0.00 & -95.25 &  & -600.00 & -620.36 &  (3 \%) & 627.63 &  (5 \%)\\
0.00 & -97.21 &  & -450.00 & -470.18 &  (4 \%) & 480.12 &  (7 \%)\\
0.00 & -98.07 &  & -300.00 & -317.89 &  (6 \%) & 332.68 &  (11 \%)\\
0.00 & -98.00 &  & -150.00 & -158.64 &  (6 \%) & 186.47 &  (24 \%)\\
0.00 & -98.60 &  & 0.00 & 14.27 &  & 99.63 & \\
0.00 & -98.50 &  & 150.00 & 176.64 &  (18 \%) & 202.25 &  (35 \%)\\
0.00 & -97.36 &  & 300.00 & 343.18 &  (14 \%) & 356.72 &  (19 \%)\\
0.00 & -94.21 &  & 450.00 & 508.79 &  (13 \%) & 517.44 &  (15 \%)\\
0.00 & -92.67 &  & 600.00 & 658.70 &  (10 \%) & 665.19 &  (11 \%)\\
0.00 & -90.95 &  & 750.00 & 803.25 &  (7 \%) & 808.38 &  (8 \%)\\
0.00 & -88.57 &  & 1000.00 & 1063.50 &  (6 \%) & 1067.18 &  (7 \%)\\
\bottomrule
\end{tabular}
\caption{ BPM data compared to CAD data for run 359711. Columns are from left to right, we see the CAD planned horizontal beam displacement, the bpm-measured horizontal beam displacement, $||CAD_{x}| - |BPM_{x}||$ (to account for polarity flips in bpm), the CAD planned vertical beam displacement, the bpm-measured beam displacement, $||CAD_{y}| - |BPM_{y}||$, the total beam separation ($\sqrt{BPM_{x}^2+BPM_{y}^2}$) and the difference between the measured total separation and the CAD planned total separation. Nominally, CAD promises to hold one beam fixed, and scan the other beam. Rows are each scan step planned and measured for the run. }
\label{0xffbafcfc}
\end{table}

\begin{table}
\centering
\begin{tabular}{c c c c c c c c}
\toprule
\textbf{$CAD_{x}$} & \textbf{$BPM_{x}$ } & \textbf{$\Delta_{x}$} &\textbf{$CAD_{y}$} & \textbf{$BPM_{y}$} & \textbf{$\Delta_{y}$} & \textbf{$BPM_{tot}$} & \textbf{$\Delta_{tot}$} \\
($10^{-5} m$) & ($10^{-5} m$) & (\%diff) & ($10^{-5} m$) & ($10^{-5} m$) & (\%diff) & ($10^{-5} m$) & (\%diff) \\
\midrule
1000.00 & -1002.45 &  (0 \%) & 0.00 & 64.83 &  & 1004.55 &  (0 \%)\\
750.00 & -758.34 &  (1 \%) & 0.00 & 64.38 &  & 761.07 &  (1 \%)\\
600.00 & -617.17 &  (3 \%) & 0.00 & 64.83 &  & 620.56 &  (3 \%)\\
450.00 & -462.77 &  (3 \%) & 0.00 & 63.90 &  & 467.16 &  (4 \%)\\
300.00 & -310.67 &  (4 \%) & 0.00 & 65.70 &  & 317.54 &  (6 \%)\\
150.00 & -150.77 &  (1 \%) & 0.00 & 64.43 &  & 163.96 &  (9 \%)\\
0.00 & 14.63 &  & 0.00 & 63.37 &  & 65.03 & \\
-150.00 & 187.85 &  (25 \%) & 0.00 & 63.54 &  & 198.30 &  (32 \%)\\
-300.00 & 349.90 &  (17 \%) & 0.00 & 63.47 &  & 355.61 &  (19 \%)\\
-450.00 & 505.68 &  (12 \%) & 0.00 & 63.04 &  & 509.59 &  (13 \%)\\
-600.00 & 654.33 &  (9 \%) & 0.00 & 64.47 &  & 657.50 &  (10 \%)\\
-750.00 & 804.53 &  (7 \%) & 0.00 & 63.97 &  & 807.07 &  (8 \%)\\
-1000.00 & 1049.00 &  (5 \%) & 0.00 & 64.92 &  & 1051.01 &  (5 \%)\\
0.00 & 28.71 &  & 1000.00 & -970.71 &  (3 \%) & 971.14 &  (3 \%)\\
0.00 & 28.00 &  & 750.00 & -729.57 &  (3 \%) & 730.10 &  (3 \%)\\
0.00 & 28.63 &  & 600.00 & -573.67 &  (4 \%) & 574.38 &  (4 \%)\\
0.00 & 29.43 &  & 450.00 & -433.07 &  (4 \%) & 434.07 &  (4 \%)\\
0.00 & 28.16 &  & 300.00 & -277.50 &  (8 \%) & 278.92 &  (7 \%)\\
0.00 & 27.40 &  & 150.00 & -105.50 &  (30 \%) & 109.00 &  (27 \%)\\
0.00 & 25.23 &  & 0.00 & 62.73 &  & 67.62 & \\
0.00 & 24.37 &  & -150.00 & 221.73 &  (48 \%) & 223.07 &  (49 \%)\\
0.00 & 25.08 &  & -300.00 & 394.88 &  (32 \%) & 395.68 &  (32 \%)\\
0.00 & 24.96 &  & -450.00 & 553.11 &  (23 \%) & 553.67 &  (23 \%)\\
0.00 & 23.71 &  & -600.00 & 701.89 &  (17 \%) & 702.29 &  (17 \%)\\
0.00 & 25.25 &  & -750.00 & 852.71 &  (14 \%) & 853.09 &  (14 \%)\\
0.00 & 23.27 &  & -1000.00 & 1105.00 &  (10 \%) & 1105.25 &  (11 \%)\\
\bottomrule
\end{tabular}
\caption{ BPM data compared to CAD data for run 360879. Columns are from left to right, we see the CAD planned horizontal beam displacement, the bpm-measured horizontal beam displacement, $||CAD_{x}| - |BPM_{x}||$ (to account for polarity flips in bpm), the CAD planned vertical beam displacement, the bpm-measured beam displacement, $||CAD_{y}| - |BPM_{y}||$, the total beam separation ($\sqrt{BPM_{x}^2+BPM_{y}^2}$) and the difference between the measured total separation and the CAD planned total separation. Nominally, CAD promises to hold one beam fixed, and scan the other beam. Rows are each scan step planned and measured for the run. }
\label{0xffb3a73c}
\end{table}

\begin{table}
\centering
\begin{tabular}{c c c c c c c c}
\toprule
\textbf{$CAD_{x}$} & \textbf{$BPM_{x}$ } & \textbf{$\Delta_{x}$} &\textbf{$CAD_{y}$} & \textbf{$BPM_{y}$} & \textbf{$\Delta_{y}$} & \textbf{$BPM_{tot}$} & \textbf{$\Delta_{tot}$} \\
($10^{-5} m$) & ($10^{-5} m$) & (\%diff) & ($10^{-5} m$) & ($10^{-5} m$) & (\%diff) & ($10^{-5} m$) & (\%diff) \\
\midrule
0.00 & 39.32 &  & 1000.00 & 1134.03 &  (13 \%) & 1134.71 &  (13 \%)\\
0.00 & 38.60 &  & 750.00 & 884.00 &  (18 \%) & 884.84 &  (18 \%)\\
0.00 & 37.77 &  & 600.00 & 735.70 &  (23 \%) & 736.67 &  (23 \%)\\
0.00 & 36.37 &  & 450.00 & 587.30 &  (31 \%) & 588.42 &  (31 \%)\\
0.00 & 36.60 &  & 300.00 & 422.33 &  (41 \%) & 423.92 &  (41 \%)\\
0.00 & 36.03 &  & 150.00 & 264.10 &  (76 \%) & 266.55 &  (78 \%)\\
0.00 & 35.36 &  & 0.00 & 98.25 &  & 104.42 & \\
0.00 & 35.54 &  & -150.00 & -79.54 &  (47 \%) & 87.11 &  (42 \%)\\
0.00 & 34.79 &  & -300.00 & -245.00 &  (18 \%) & 247.46 &  (18 \%)\\
0.00 & 33.82 &  & -450.00 & -399.00 &  (11 \%) & 400.43 &  (11 \%)\\
0.00 & 33.83 &  & -600.00 & -555.93 &  (7 \%) & 556.96 &  (7 \%)\\
0.00 & 32.50 &  & -750.00 & -705.93 &  (6 \%) & 706.68 &  (6 \%)\\
0.00 & 32.25 &  & -1000.00 & -958.38 &  (4 \%) & 958.92 &  (4 \%)\\
1000.00 & 1074.84 &  (7 \%) & 0.00 & 104.80 &  & 1079.94 &  (8 \%)\\
750.00 & 832.50 &  (11 \%) & 0.00 & 102.56 &  & 838.79 &  (12 \%)\\
600.00 & 680.97 &  (13 \%) & 0.00 & 100.27 &  & 688.31 &  (15 \%)\\
450.00 & 528.89 &  (18 \%) & 0.00 & 97.68 &  & 537.84 &  (20 \%)\\
300.00 & 367.29 &  (22 \%) & 0.00 & 95.75 &  & 379.56 &  (27 \%)\\
150.00 & 215.44 &  (44 \%) & 0.00 & 92.94 &  & 234.63 &  (56 \%)\\
0.00 & 43.07 &  & 0.00 & 90.23 &  & 99.98 & \\
-150.00 & -123.21 &  (18 \%) & 0.00 & 89.32 &  & 152.18 &  (1 \%)\\
-300.00 & -293.20 &  (2 \%) & 0.00 & 88.00 &  & 306.12 &  (2 \%)\\
-450.00 & -440.63 &  (2 \%) & 0.00 & 85.10 &  & 448.78 &  (0 \%)\\
-600.00 & -597.93 &  (0 \%) & 0.00 & 83.50 &  & 603.74 &  (1 \%)\\
-750.00 & -744.61 &  (1 \%) & 0.00 & 82.71 &  & 749.19 &  (0 \%)\\
-1000.00 & -990.64 &  (1 \%) & 0.00 & 79.35 &  & 993.81 &  (1 \%)\\
\bottomrule
\end{tabular}
\caption{ BPM data compared to CAD data for run 362492. Columns are from left to right, we see the CAD planned horizontal beam displacement, the bpm-measured horizontal beam displacement, $||CAD_{x}| - |BPM_{x}||$ (to account for polarity flips in bpm), the CAD planned vertical beam displacement, the bpm-measured beam displacement, $||CAD_{y}| - |BPM_{y}||$, the total beam separation ($\sqrt{BPM_{x}^2+BPM_{y}^2}$) and the difference between the measured total separation and the CAD planned total separation. Nominally, CAD promises to hold one beam fixed, and scan the other beam. Rows are each scan step planned and measured for the run. }
\label{0xff85589c}
\end{table}

\begin{table}
\centering
\begin{tabular}{c c c c c c c c}
\toprule
\textbf{$CAD_{x}$} & \textbf{$BPM_{x}$ } & \textbf{$\Delta_{x}$} &\textbf{$CAD_{y}$} & \textbf{$BPM_{y}$} & \textbf{$\Delta_{y}$} & \textbf{$BPM_{tot}$} & \textbf{$\Delta_{tot}$} \\
($10^{-5} m$) & ($10^{-5} m$) & (\%diff) & ($10^{-5} m$) & ($10^{-5} m$) & (\%diff) & ($10^{-5} m$) & (\%diff) \\
\midrule
200.00 & -183.60 &  (8 \%) & 0.00 & 22.50 &  & 184.97 &  (8 \%)\\
400.00 & -382.50 &  (4 \%) & 0.00 & 23.50 &  & 383.22 &  (4 \%)\\
600.00 & -581.25 &  (3 \%) & 0.00 & 23.33 &  & 581.72 &  (3 \%)\\
800.00 & -777.70 &  (3 \%) & 0.00 & 23.80 &  & 778.06 &  (3 \%)\\
0.00 & 41.30 &  & 0.00 & 25.40 &  & 48.49 & \\
-200.00 & 267.83 &  (34 \%) & 0.00 & 23.17 &  & 268.83 &  (34 \%)\\
-400.00 & 467.83 &  (17 \%) & 0.00 & 23.00 &  & 468.40 &  (17 \%)\\
-600.00 & 667.83 &  (11 \%) & 0.00 & 23.42 &  & 668.24 &  (11 \%)\\
-800.00 & 863.17 &  (8 \%) & 0.00 & 24.00 &  & 863.50 &  (8 \%)\\
0.00 & 46.70 &  & 200.00 & -201.50 &  (1 \%) & 206.84 &  (3 \%)\\
0.00 & 47.92 &  & 400.00 & -399.75 &  (0 \%) & 402.61 &  (1 \%)\\
0.00 & 48.58 &  & 600.00 & -600.50 &  (0 \%) & 602.46 &  (0 \%)\\
0.00 & 49.60 &  & 800.00 & -797.20 &  (0 \%) & 798.74 &  (0 \%)\\
0.00 & 47.10 &  & 0.00 & 26.00 &  & 53.80 & \\
0.00 & 46.42 &  & -200.00 & 248.58 &  (24 \%) & 252.88 &  (26 \%)\\
0.00 & 45.67 &  & -400.00 & 455.33 &  (14 \%) & 457.62 &  (14 \%)\\
0.00 & 45.33 &  & -600.00 & 652.75 &  (9 \%) & 654.32 &  (9 \%)\\
0.00 & 44.92 &  & -800.00 & 851.08 &  (6 \%) & 852.27 &  (7 \%)\\
\bottomrule
\end{tabular}
\caption{ BPM data compared to CAD data for run 364636. Columns are from left to right, we see the CAD planned horizontal beam displacement, the bpm-measured horizontal beam displacement, $||CAD_{x}| - |BPM_{x}||$ (to account for polarity flips in bpm), the CAD planned vertical beam displacement, the bpm-measured beam displacement, $||CAD_{y}| - |BPM_{y}||$, the total beam separation ($\sqrt{BPM_{x}^2+BPM_{y}^2}$) and the difference between the measured total separation and the CAD planned total separation. Nominally, CAD promises to hold one beam fixed, and scan the other beam. Rows are each scan step planned and measured for the run. }
\label{0xff9ededc}
\end{table}

\begin{table}
\centering
\begin{tabular}{c c c c c c c c}
\toprule
\textbf{$CAD_{x}$} & \textbf{$BPM_{x}$ } & \textbf{$\Delta_{x}$} &\textbf{$CAD_{y}$} & \textbf{$BPM_{y}$} & \textbf{$\Delta_{y}$} & \textbf{$BPM_{tot}$} & \textbf{$\Delta_{tot}$} \\
($10^{-5} m$) & ($10^{-5} m$) & (\%diff) & ($10^{-5} m$) & ($10^{-5} m$) & (\%diff) & ($10^{-5} m$) & (\%diff) \\
\midrule
-600.00 & -577.90 &  (4 \%) & 0.00 & -6.73 &  & 577.94 &  (4 \%)\\
-400.00 & -385.42 &  (4 \%) & 0.00 & -3.75 &  & 385.43 &  (4 \%)\\
-320.00 & -305.67 &  (4 \%) & 0.00 & -2.68 &  & 305.68 &  (4 \%)\\
-240.00 & -226.53 &  (6 \%) & 0.00 & -1.53 &  & 226.53 &  (6 \%)\\
-160.00 & -140.53 &  (12 \%) & 0.00 & -0.53 &  & 140.53 &  (12 \%)\\
-80.00 & -50.83 &  (36 \%) & 0.00 & 1.10 &  & 50.85 &  (36 \%)\\
0.00 & 38.33 &  & 0.00 & 3.87 &  & 38.53 & \\
80.00 & 134.70 &  (68 \%) & 0.00 & 4.23 &  & 134.77 &  (68 \%)\\
160.00 & 224.25 &  (40 \%) & 0.00 & 5.25 &  & 224.31 &  (40 \%)\\
240.00 & 308.87 &  (29 \%) & 0.00 & 6.00 &  & 308.93 &  (29 \%)\\
320.00 & 391.42 &  (22 \%) & 0.00 & 7.13 &  & 391.49 &  (22 \%)\\
400.00 & 469.58 &  (17 \%) & 0.00 & 8.22 &  & 469.66 &  (17 \%)\\
600.00 & 665.71 &  (11 \%) & 0.00 & 10.44 &  & 665.79 &  (11 \%)\\
0.00 & 39.85 &  & -600.00 & -629.28 &  (5 \%) & 630.54 &  (5 \%)\\
0.00 & 39.78 &  & -400.00 & -431.41 &  (8 \%) & 433.24 &  (8 \%)\\
0.00 & 40.08 &  & -320.00 & -354.02 &  (11 \%) & 356.29 &  (11 \%)\\
0.00 & 40.55 &  & -240.00 & -269.69 &  (12 \%) & 272.73 &  (14 \%)\\
0.00 & 41.57 &  & -160.00 & -184.77 &  (15 \%) & 189.38 &  (18 \%)\\
0.00 & 41.75 &  & -80.00 & -93.50 &  (17 \%) & 102.40 &  (28 \%)\\
0.00 & 41.88 &  & 0.00 & 1.69 &  & 41.91 & \\
0.00 & 43.00 &  & 80.00 & 90.91 &  (14 \%) & 100.56 &  (26 \%)\\
0.00 & 42.57 &  & 160.00 & 184.73 &  (15 \%) & 189.57 &  (18 \%)\\
0.00 & 41.21 &  & 240.00 & 271.82 &  (13 \%) & 274.92 &  (15 \%)\\
0.00 & 41.11 &  & 320.00 & 356.32 &  (11 \%) & 358.68 &  (12 \%)\\
0.00 & 41.69 &  & 400.00 & 437.19 &  (9 \%) & 439.18 &  (10 \%)\\
0.00 & 42.33 &  & 600.00 & 636.48 &  (6 \%) & 637.88 &  (6 \%)\\
\bottomrule
\end{tabular}
\caption{ BPM data compared to CAD data for run 365866. Columns are from left to right, we see the CAD planned horizontal beam displacement, the bpm-measured horizontal beam displacement, $||CAD_{x}| - |BPM_{x}||$ (to account for polarity flips in bpm), the CAD planned vertical beam displacement, the bpm-measured beam displacement, $||CAD_{y}| - |BPM_{y}||$, the total beam separation ($\sqrt{BPM_{x}^2+BPM_{y}^2}$) and the difference between the measured total separation and the CAD planned total separation. Nominally, CAD promises to hold one beam fixed, and scan the other beam. Rows are each scan step planned and measured for the run. }
\label{0xffd6370c}
\end{table}

\begin{table}
\centering
\begin{tabular}{c c c c c c c c}
\toprule
\textbf{$CAD_{x}$} & \textbf{$BPM_{x}$ } & \textbf{$\Delta_{x}$} &\textbf{$CAD_{y}$} & \textbf{$BPM_{y}$} & \textbf{$\Delta_{y}$} & \textbf{$BPM_{tot}$} & \textbf{$\Delta_{tot}$} \\
($10^{-5} m$) & ($10^{-5} m$) & (\%diff) & ($10^{-5} m$) & ($10^{-5} m$) & (\%diff) & ($10^{-5} m$) & (\%diff) \\
\midrule
-600.00 & 677.17 &  (13 \%) & 0.00 & 5.79 &  & 677.19 &  (13 \%)\\
-400.00 & 481.89 &  (20 \%) & 0.00 & 6.56 &  & 481.93 &  (20 \%)\\
-320.00 & 408.56 &  (28 \%) & 0.00 & 7.94 &  & 408.64 &  (28 \%)\\
-240.00 & 330.50 &  (38 \%) & 0.00 & 7.45 &  & 330.58 &  (38 \%)\\
-160.00 & 243.18 &  (52 \%) & 0.00 & 6.07 &  & 243.25 &  (52 \%)\\
-80.00 & 150.77 &  (88 \%) & 0.00 & 4.47 &  & 150.83 &  (89 \%)\\
0.00 & 56.41 &  & 0.00 & 4.85 &  & 56.62 & \\
80.00 & -38.10 &  (52 \%) & 0.00 & 4.97 &  & 38.42 &  (52 \%)\\
160.00 & -129.58 &  (19 \%) & 0.00 & 4.68 &  & 129.66 &  (19 \%)\\
240.00 & -213.16 &  (11 \%) & 0.00 & 5.05 &  & 213.22 &  (11 \%)\\
320.00 & -292.25 &  (9 \%) & 0.00 & 7.75 &  & 292.35 &  (9 \%)\\
400.00 & -373.50 &  (7 \%) & 0.00 & 5.88 &  & 373.55 &  (7 \%)\\
600.00 & -571.65 &  (5 \%) & 0.00 & 6.93 &  & 571.69 &  (5 \%)\\
0.00 & 48.52 &  & -600.00 & 631.83 &  (5 \%) & 633.69 &  (6 \%)\\
0.00 & 51.12 &  & -400.00 & 437.19 &  (9 \%) & 440.17 &  (10 \%)\\
0.00 & 51.72 &  & -320.00 & 360.19 &  (13 \%) & 363.89 &  (14 \%)\\
0.00 & 52.21 &  & -240.00 & 278.79 &  (16 \%) & 283.64 &  (18 \%)\\
0.00 & 51.40 &  & -160.00 & 195.13 &  (22 \%) & 201.79 &  (26 \%)\\
0.00 & 50.93 &  & -80.00 & 104.20 &  (30 \%) & 115.98 &  (45 \%)\\
0.00 & 50.89 &  & 0.00 & 10.46 &  & 51.96 & \\
0.00 & 50.97 &  & 80.00 & -84.63 &  (6 \%) & 98.79 &  (23 \%)\\
0.00 & 51.15 &  & 160.00 & -173.54 &  (8 \%) & 180.92 &  (13 \%)\\
0.00 & 52.15 &  & 240.00 & -258.91 &  (8 \%) & 264.11 &  (10 \%)\\
0.00 & 52.56 &  & 320.00 & -341.56 &  (7 \%) & 345.58 &  (8 \%)\\
0.00 & 53.44 &  & 400.00 & -416.78 &  (4 \%) & 420.19 &  (5 \%)\\
0.00 & 54.71 &  & 600.00 & -616.50 &  (3 \%) & 618.92 &  (3 \%)\\
\bottomrule
\end{tabular}
\caption{ BPM data compared to CAD data for run 366605. Columns are from left to right, we see the CAD planned horizontal beam displacement, the bpm-measured horizontal beam displacement, $||CAD_{x}| - |BPM_{x}||$ (to account for polarity flips in bpm), the CAD planned vertical beam displacement, the bpm-measured beam displacement, $||CAD_{y}| - |BPM_{y}||$, the total beam separation ($\sqrt{BPM_{x}^2+BPM_{y}^2}$) and the difference between the measured total separation and the CAD planned total separation. Nominally, CAD promises to hold one beam fixed, and scan the other beam. Rows are each scan step planned and measured for the run. }
\label{0xffdb500c}
\end{table}

\begin{table}
\centering
\begin{tabular}{c c c c c c c c}
\toprule
\textbf{$CAD_{x}$} & \textbf{$BPM_{x}$ } & \textbf{$\Delta_{x}$} &\textbf{$CAD_{y}$} & \textbf{$BPM_{y}$} & \textbf{$\Delta_{y}$} & \textbf{$BPM_{tot}$} & \textbf{$\Delta_{tot}$} \\
($10^{-5} m$) & ($10^{-5} m$) & (\%diff) & ($10^{-5} m$) & ($10^{-5} m$) & (\%diff) & ($10^{-5} m$) & (\%diff) \\
\midrule
-600.00 & -588.44 &  (2 \%) & 0.00 & -3.98 &  & 588.45 &  (2 \%)\\
-400.00 & -396.78 &  (1 \%) & 0.00 & -0.12 &  & 396.78 &  (1 \%)\\
-320.00 & -319.86 &  (0 \%) & 0.00 & -0.67 &  & 319.86 &  (0 \%)\\
-240.00 & -242.17 &  (1 \%) & 0.00 & 1.89 &  & 242.17 &  (1 \%)\\
-160.00 & -158.47 &  (1 \%) & 0.00 & 2.88 &  & 158.50 &  (1 \%)\\
-80.00 & -63.33 &  (21 \%) & 0.00 & 3.88 &  & 63.45 &  (21 \%)\\
0.00 & 35.20 &  & 0.00 & 4.53 &  & 35.49 & \\
80.00 & 135.30 &  (69 \%) & 0.00 & 4.97 &  & 135.39 &  (69 \%)\\
160.00 & 229.44 &  (43 \%) & 0.00 & 6.00 &  & 229.52 &  (43 \%)\\
240.00 & 315.29 &  (31 \%) & 0.00 & 7.91 &  & 315.39 &  (31 \%)\\
320.00 & 396.29 &  (24 \%) & 0.00 & 7.32 &  & 396.36 &  (24 \%)\\
400.00 & 471.53 &  (18 \%) & 0.00 & 7.69 &  & 471.59 &  (18 \%)\\
600.00 & 666.17 &  (11 \%) & 0.00 & 10.67 &  & 666.25 &  (11 \%)\\
0.00 & 38.91 &  & -600.00 & -635.34 &  (6 \%) & 636.53 &  (6 \%)\\
0.00 & 38.59 &  & -400.00 & -438.25 &  (10 \%) & 439.95 &  (10 \%)\\
0.00 & 38.75 &  & -320.00 & -359.68 &  (12 \%) & 361.76 &  (13 \%)\\
0.00 & 39.08 &  & -240.00 & -277.36 &  (16 \%) & 280.10 &  (17 \%)\\
0.00 & 38.77 &  & -160.00 & -193.11 &  (21 \%) & 196.96 &  (23 \%)\\
0.00 & 39.17 &  & -80.00 & -96.79 &  (21 \%) & 104.41 &  (31 \%)\\
0.00 & 41.33 &  & 0.00 & 3.43 &  & 41.48 & \\
0.00 & 41.79 &  & 80.00 & 106.54 &  (33 \%) & 114.44 &  (43 \%)\\
0.00 & 42.97 &  & 160.00 & 196.62 &  (23 \%) & 201.27 &  (26 \%)\\
0.00 & 42.50 &  & 240.00 & 282.81 &  (18 \%) & 285.99 &  (19 \%)\\
0.00 & 41.45 &  & 320.00 & 364.29 &  (14 \%) & 366.64 &  (15 \%)\\
0.00 & 41.41 &  & 400.00 & 442.75 &  (11 \%) & 444.68 &  (11 \%)\\
0.00 & 41.38 &  & 600.00 & 642.98 &  (7 \%) & 644.31 &  (7 \%)\\
\bottomrule
\end{tabular}
\caption{ BPM data compared to CAD data for run 367138. Columns are from left to right, we see the CAD planned horizontal beam displacement, the bpm-measured horizontal beam displacement, $||CAD_{x}| - |BPM_{x}||$ (to account for polarity flips in bpm), the CAD planned vertical beam displacement, the bpm-measured beam displacement, $||CAD_{y}| - |BPM_{y}||$, the total beam separation ($\sqrt{BPM_{x}^2+BPM_{y}^2}$) and the difference between the measured total separation and the CAD planned total separation. Nominally, CAD promises to hold one beam fixed, and scan the other beam. Rows are each scan step planned and measured for the run. }
\label{0xff9f47dc}
\end{table}


