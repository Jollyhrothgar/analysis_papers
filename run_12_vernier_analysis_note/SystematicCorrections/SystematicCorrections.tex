
%%%%%%%%%%%%%%%%%%%%%%%%%%%%%%%%%%%%%%%%%%%%%%%%%%%%%%%%%%%%%%%%%
%                                                               %
% Chapter: SystematicCorrections Wed Sep  9 22:48:07 EDT 2015
%                                                               %
%%%%%%%%%%%%%%%%%%%%%%%%%%%%%%%%%%%%%%%%%%%%%%%%%%%%%%%%%%%%%%%%%

\chapter{Systematic Corrections}
\label{ch:SystematicCorrections}

This chapter serves as an introduction to the various corrections which are
applied along the analysis chain. In the case where corrections are small, and
don't warrant an entire chapter, I will discuss in detail here, but for other
correcitons, I will simply introduce here, and devote a whole chapter to them.

\section{Crossing Angle and $\beta^{*}$: The Hourglass Effect}

In the model for beam lumonisty discussed in Equation ~\ref{eq:lumi_one_bunch},
section ~\ref{ch:intro}(BROKEN REFERENCE), we neglected to  mention the importance of the
z-dependant ion distributions in each bunch. In fact, beam bunches have a very
real z-dependant componant of their ion distributions. If we could view the
transverse beam width at various points along the bunch in the z-dimension, we
would find that the transverse beam width varied as a function of z. However,
because we must trigger our data, our mininmum bias trigger imposes vertex range
over which we can observe events.  When we measure horizontal and vertical beam
widths using the vernier scan technique, we are effectively measuring weighted
average horizontal and vertical beam widths over a z-vertex range of 0 to 30
centimeters ~\cite{an888}. Suffice to say, we have an incomplete model for our
beam gemoetry, and must account for the z-dependance of our bunch density
distributions, and we do so using the measured horizontal and vertical beam
widths as a constraint. 

Chapter~\ref{ch:HourglassCorrection} discusses this correction in detail.

\section{Luminosity Losses}

Anyone who has been on shift is familiar with the overhead displays of the beam
intensity. We all know that as any fill progresses, there is luminosity loss,
which will affect the BBC Rate. Losses in the BBC Rate will skew any
distributions that depend on this parameter, and in this analysis, the beam
width is such a distribution. Vernier scans are taken towards the end of a fill,
since luminosity losses are more extreme at the beginning of the fill. In
general, these losses account for about 1 or 2 percent, though they are
calculated explicitly in this analysis over the duration of each scan. 

{\textbf{NOTE: show how we derive a linear relationship between beam losses and the
product of the WCMBlueTotal and WCMYellowTotal, fit this with a line, and then
use the fit to correct the BBC rate data}

\section{Beam Position Monitoring vs Planned Scan Steps}

\section{Time Synchronization}

