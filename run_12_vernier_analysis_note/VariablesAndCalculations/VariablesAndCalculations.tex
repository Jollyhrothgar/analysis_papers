%%%%%%%%%%%%%%%%%%%%%%%%%%%%%%%%%%%%%%%%%%%%%%%%%%%%%%%%%%%%%%%%%
%                                                               %
% Chapter: VariablesAndCalculations Thu Sep  3 22:23:57 EDT 2015
%                                                               %
%%%%%%%%%%%%%%%%%%%%%%%%%%%%%%%%%%%%%%%%%%%%%%%%%%%%%%%%%%%%%%%%%

\chapter{Variables and Calculations}
\label{ch:VariablesAndCalculations}
Though the process of extracting the variables presented in this chapter can sometimes
require several steps (or even its own chapter) we present the variables in this chapter
from a physical standpoint. 

These variables characterize the dynamics of the beams intersecting in the PHENIX
interaction region (IR). Our goal in the vernier analysis is to calculate the effective
detector cross section for our Beam Beam Counter (BBC) photomultiplier tube detector, as
well as the RHIC machine luminosity, $\mathcal{L}_{RHIC}$. We must also use a realistic
model for highly relativistic collisions between two intersecting beams.

The basic equations, models and variables used in the analysis are summarized here; no
derivations will be provided (but they will be referenced for the interested reader). I
will go into detail discussing the extraciton of each parameter within relevent chapters.

The effective detector cross section~\cite{drees2013} can be expressed as:

\begin{equation}
\label{eq:detector_xsec}
\sigma_{ZDC}^{eff} = {{R_{max} 2\pi n_{B} n_{Y} \sigma_{x} \sigma_{y}}\over{n_{bunch}
f_{bunch} N_{B} N_{Y} }}
\end{equation}

with various parameters are defined in Table~\ref{tab:ana_vars}.

We use the standard relativistic intersecting beam model for colliding
bunches~\cite{an888}, which is defined to be:

\begin{equation}
\label{eq:reletivistic_beam_final} 
\mathcal{L} = {{n_{bunch} f_{bunch} N_{B} N_{Y} }\over{ 2 \pi^{2} \sigma_{x}
\sigma_{y}\sigma{z}^2}} \int \int
e^{-\left({{z^{2}}\over{\sigma_{z}^{2}}}+{{c^{2}t^{2}}\over{\sigma_{z}^{2}}}\right)} c dt
\end{equation}

\begin{table}
\centering
\begin{tabular}{c p{8cm} c c }
\toprule
\textbf{Variable} & \textbf{Description} & \textbf{Units} & \textbf{Dimension} \\
\midrule 
$\sigma_{x}$, $\sigma_{y}$ & bunch profile width obtained from vernier scan beam overlap in x and y directions. & $\mu m$ & $length$ \\
$\sigma_{z}$ & bunch profile width in z-direction, i.e. z-beam width. & $\mu m$ & $length$ \\
$\dot{N}$ & Live event rate for "BBCLL1(\textgreater0 tubes)" trigger & $Hz$ & $time^{-1}$ \\
$R_{max}$ & Maximum BBC Rate determined from maximum overlap of beams rate for "BBCLL1(\textgreater0 tubes)" trigger & $Hz$ & $time^{-1}$ \\
$\mathcal{L}_{RHIC}$ & Absolute luminosity delivered to PHENIX IR from RHIC & $cm^{-2}s^{-1}$ & $area^{-1}time^{1}$ \\
$\sigma_{p+p} $ & Inelastic scattering cross section of proton-proton collisions & $cm^{2}$ & $area^{-1}$ \\
$N_{b}^{i},N_{y}^{i}$ & Number of ions in bunch $i$ for the blue ($b$) beam or yellow ($y$) beam. & count & \\
$f_{bunch}$ & Frequency of a specific bunch crossing & $Hz$ & $seconds^{-1}$ \\
$k_{b}$ & Number of bunches filled in one of the beams (assume identical beams) & count & \\
$\sigma_{BBC}$ & Cross section of p+p collisions observed by BBC, uncorrected for efficiency & $cm^{-2}$ & $area_{-1}$ \\
$\epsilon_{BBC}$ & BBC efficiency & unitless $[0,1]$ & \\
$n_{bunch}$ & Number of filled bunches in the blue or yellow beam & unitless $[0,120]$ & \\
\bottomrule
\end{tabular}
\caption{These are the variables we use in the vernier analysis. Some variables are
extracted directly from the data streams (such as the bbc-rate), while others are
calculated from distributions of variables (such as the beam-width, $\sigma_{x,y}$). }
\label{tab:ana_vars}
\end{table}

